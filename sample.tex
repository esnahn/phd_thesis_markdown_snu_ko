% 서울대학교 전기공학부 (전기컴퓨터공학부) 석사ㅡ 박사 학위논문
% LaTeX 양식 샘플
\RequirePackage{fix-cm} % documentclass 이전에 넣는다.
% oneside : 단면 인쇄용
% twoside : 양면 인쇄용
% ko : 국문 논문 작성
% master : 석사
% phd : 박사
% openright : 챕터가 홀수쪽에서 시작
\documentclass[oneside,phd,]{snuthesis}

%%%%%%%%%%%%%%%%%%%%%%%%%%%%%%%%%%%%%%%%
%% 목차 양식을 변경하는 코드
%% subfigure (subfig) package 사용 여부에 따라
%% tocloft의 옵션을 다르게 지정해야 한다.
%\usepackage[titles,subfigure]{tocloft} % when you use subfigure package
\usepackage[titles]{tocloft} % when you don't use subfigure package
\usepackage{calc}
\makeatletter % don't delete me
\if@snu@ko
	\renewcommand\cftchappresnum{제~}
	\renewcommand\cftchapaftersnum{~장}
	\renewcommand\cftfigpresnum{그림~}
	\renewcommand\cfttabpresnum{표~}
\else
	\renewcommand\cftchappresnum{Chapter~}
	\renewcommand\cftfigpresnum{Figure~}
	\renewcommand\cfttabpresnum{Table~}
\fi
\makeatother % don't delete me
\newlength{\mytmplen}
\settowidth{\mytmplen}{\bfseries\cftchappresnum\cftchapaftersnum}
\addtolength{\cftchapnumwidth}{\mytmplen}
\settowidth{\mytmplen}{\bfseries\cftfigpresnum\cftfigaftersnum}
\addtolength{\cftfignumwidth}{\mytmplen}
\settowidth{\mytmplen}{\bfseries\cfttabpresnum\cfttabaftersnum}
\addtolength{\cfttabnumwidth}{\mytmplen}

\renewcommand{\cftchappresnum}{\chaptername\space}
\setlength{\cftchapnumwidth}{\widthof{\textbf{Appendix~999~}}}
\makeatletter
\g@addto@macro\appendix{%
  \addtocontents{toc}{%
    \protect\renewcommand{\protect\cftchappresnum}{\appendixname\space}%
  }%
}
%% 목차 양식을 변경하는 코드 끝
%%%%%%%%%%%%%%%%%%%%%%%%%%%%%%%%%%%%%%%%

%%%%%%%%%%%%%%%%%%%%%%%%%%%%%%%%%%%%%%%%
%% 다른 패키지 로드
%% http://faq.ktug.or.kr/faq/pdflatex%B0%FAlatex%B5%BF%BD%C3%BB%E7%BF%EB
%% 필요에 따라 직접 수정 필요
\ifpdf
	\input glyphtounicode\pdfgentounicode=1 %type 1 font사용시
	%\usepackage[pdftex,unicode]{hyperref} % delete me
	\usepackage[pdftex]{graphicx}
  \usepackage{setspace}
	%\usepackage[pdftex,svgnames]{xcolor}
\else
	%\usepackage[dvipdfmx,unicode]{hyperref} % delete me
	\usepackage[dvipdfmx]{graphicx}
	%\usepackage[dvipdfmx,svgnames]{xcolor}
\fi









\usepackage[T1]{fontenc}
\usepackage{lmodern}
\usepackage{amssymb,amsmath}
\usepackage{ifxetex,ifluatex}
\usepackage{fixltx2e} % provides \textsubscript
% use upquote if available, for straight quotes in verbatim environments
\IfFileExists{upquote.sty}{\usepackage{upquote}}{}

% use microtype if available
\IfFileExists{microtype.sty}{\usepackage{microtype}}{}
\usepackage{listings}
\usepackage{color}
\usepackage{fancyvrb}
\newcommand{\VerbBar}{|}
\newcommand{\VERB}{\Verb[commandchars=\\\{\}]}
\DefineVerbatimEnvironment{Highlighting}{Verbatim}{commandchars=\\\{\}}
% Add ',fontsize=\small' for more characters per line
\usepackage{framed}
\definecolor{shadecolor}{RGB}{248,248,248}
\newenvironment{Shaded}{\begin{snugshade}}{\end{snugshade}}
\newcommand{\KeywordTok}[1]{\textcolor[rgb]{0.13,0.29,0.53}{\textbf{{#1}}}}
\newcommand{\DataTypeTok}[1]{\textcolor[rgb]{0.13,0.29,0.53}{{#1}}}
\newcommand{\DecValTok}[1]{\textcolor[rgb]{0.00,0.00,0.81}{{#1}}}
\newcommand{\BaseNTok}[1]{\textcolor[rgb]{0.00,0.00,0.81}{{#1}}}
\newcommand{\FloatTok}[1]{\textcolor[rgb]{0.00,0.00,0.81}{{#1}}}
\newcommand{\ConstantTok}[1]{\textcolor[rgb]{0.00,0.00,0.00}{{#1}}}
\newcommand{\CharTok}[1]{\textcolor[rgb]{0.31,0.60,0.02}{{#1}}}
\newcommand{\SpecialCharTok}[1]{\textcolor[rgb]{0.00,0.00,0.00}{{#1}}}
\newcommand{\StringTok}[1]{\textcolor[rgb]{0.31,0.60,0.02}{{#1}}}
\newcommand{\VerbatimStringTok}[1]{\textcolor[rgb]{0.31,0.60,0.02}{{#1}}}
\newcommand{\SpecialStringTok}[1]{\textcolor[rgb]{0.31,0.60,0.02}{{#1}}}
\newcommand{\ImportTok}[1]{{#1}}
\newcommand{\CommentTok}[1]{\textcolor[rgb]{0.56,0.35,0.01}{\textit{{#1}}}}
\newcommand{\DocumentationTok}[1]{\textcolor[rgb]{0.56,0.35,0.01}{\textbf{\textit{{#1}}}}}
\newcommand{\AnnotationTok}[1]{\textcolor[rgb]{0.56,0.35,0.01}{\textbf{\textit{{#1}}}}}
\newcommand{\CommentVarTok}[1]{\textcolor[rgb]{0.56,0.35,0.01}{\textbf{\textit{{#1}}}}}
\newcommand{\OtherTok}[1]{\textcolor[rgb]{0.56,0.35,0.01}{{#1}}}
\newcommand{\FunctionTok}[1]{\textcolor[rgb]{0.00,0.00,0.00}{{#1}}}
\newcommand{\VariableTok}[1]{\textcolor[rgb]{0.00,0.00,0.00}{{#1}}}
\newcommand{\ControlFlowTok}[1]{\textcolor[rgb]{0.13,0.29,0.53}{\textbf{{#1}}}}
\newcommand{\OperatorTok}[1]{\textcolor[rgb]{0.81,0.36,0.00}{\textbf{{#1}}}}
\newcommand{\BuiltInTok}[1]{{#1}}
\newcommand{\ExtensionTok}[1]{{#1}}
\newcommand{\PreprocessorTok}[1]{\textcolor[rgb]{0.56,0.35,0.01}{\textit{{#1}}}}
\newcommand{\AttributeTok}[1]{\textcolor[rgb]{0.77,0.63,0.00}{{#1}}}
\newcommand{\RegionMarkerTok}[1]{{#1}}
\newcommand{\InformationTok}[1]{\textcolor[rgb]{0.56,0.35,0.01}{\textbf{\textit{{#1}}}}}
\newcommand{\WarningTok}[1]{\textcolor[rgb]{0.56,0.35,0.01}{\textbf{\textit{{#1}}}}}
\newcommand{\AlertTok}[1]{\textcolor[rgb]{0.94,0.16,0.16}{{#1}}}
\newcommand{\ErrorTok}[1]{\textcolor[rgb]{0.64,0.00,0.00}{\textbf{{#1}}}}
\newcommand{\NormalTok}[1]{{#1}}
\usepackage{longtable,booktabs}
\usepackage{graphicx}
% Redefine \includegraphics so that, unless explicit options are
% given, the image width will not exceed the width of the page.
% Images get their normal width if they fit onto the page, but
% are scaled down if they would overflow the margins.
\makeatletter
\def\ScaleIfNeeded{%
  \ifdim\Gin@nat@width>\linewidth
    \linewidth
  \else
    \Gin@nat@width
  \fi
}
\makeatother
\let\Oldincludegraphics\includegraphics
{%
 \catcode`\@=11\relax%
 \gdef\includegraphics{\@ifnextchar[{\Oldincludegraphics}{\Oldincludegraphics[width=\ScaleIfNeeded]}}%
}%
\ifxetex
  \usepackage[setpagesize=false, % page size defined by xetex
              unicode=false, % unicode breaks when used with xetex
              xetex]{hyperref}
\else
  \usepackage[unicode=true]{hyperref}
\fi
\hypersetup{breaklinks=true,
            bookmarks=true,
            %pdfauthor={홍길동},
            %pdftitle={},
            %colorlinks=true,
            %citecolor=blue,
            %urlcolor=blue,
            %linkcolor=magenta,
            %pdfborder={0 0 0}}
            }
\urlstyle{same}  % don't use monospace font for urls
\usepackage[normalem]{ulem}
% avoid problems with \sout in headers with hyperref:
\pdfstringdefDisableCommands{\renewcommand{\sout}{}}
\setlength{\parindent}{0pt}
\setlength{\parskip}{6pt plus 2pt minus 1pt}
\setlength{\emergencystretch}{3em}  % prevent overfull lines
\setcounter{secnumdepth}{0}
\VerbatimFootnotes % allows verbatim text in footnotes
\usepackage{csquotes}





\usepackage{natbib}
\bibliographystyle{apalike}
\def\cite{\citep}
% for pandoc
\providecommand{\tightlist}{%
  \setlength{\itemsep}{0pt}\setlength{\parskip}{0pt}}
%%%%%%%%%%%%%%%%%%%%%%%%%%%%%%%%%%%%%%%%


%% \title : 22pt로 나오는 큰 제목
%% \title* : 16pt로 나오는 작은 제목
\title{Calculus, Analysis, Mathematical Statistics, ~Statistical Computinig and
Lab; Regression Analysis}
\title*{미적분학 해석학 수리통계학 확률의 개념 및 응용, ~전산통계 및 실험 그리고
회귀분석 및 실험}

%% 저자 이름 Author's(Your) name
%\author{홍길동}
%\author*{홍~길~동} % Insert space for Hangul name.
\author{홍길동}
\author*{홍 길 동} % Same as \author.

%% 학번 Student number
\studentnumber{2014-12345}

%% 지도교수님 성함 Advisor's name
%% (?) Use Korean name for Korean professor.
%\advisor{홍길동}
%\advisor*{홍~길~동} % Insert space for Hangul name.
\advisor{이 교 수}
\advisor*{이 교 수}

%% 학위 수여일 Graduation date
%% 표지에 적히는 날짜.
%% 학위 수여일이 아니라 논문 발간년도를 적어야 할 수도 있음.
%\graddate{2010~년~2~월}
%\graddate{FEBRURARY 2010}
\graddate{2018 년 2 월}
%% 논문 제출일 Submission date
%% (?) Use Korean date format.
%\submissiondate{2009~년~11~월}
\submissiondate{2017 년 11 월}

%% 논문 인준일 Approval date
%% (?) Use Korean date format.
%\approvaldate{2009~년~12~월}
\approvaldate{2017 년 12 월}

%% Note: 인준지의 교수님 성함은
%% 컴퓨터로 출력하지 않고, 교수님께서
%% 자필로 쓰시기도 합니다.
%% Committee members' names
\committeemembers%
{Freddie Mercury}{이 교 수}{Brian May}{Roger Taylor}{John Deacon}
%{김교수}%
%{이교수}%
%{박교수}%
%{최교수}%
%{John Smith}
%% Length of underline
%\setlength{\committeenameunderlinelength}{7cm}

\begin{document}
\pagenumbering{Roman}
\makefrontcover
\makefrontcover
\makeapproval

\cleardoublepage
\pagenumbering{roman}
% 초록 Abstract

\keyword{Alpha, Bravo, Charlie, Delta}
\begin{abstract}

\begin{center}
\fontsize{22pt}{20pt} \selectfont
\textbf{
Calculus, Analysis, Mathematical Statistics, ~Statistical Computinig and
Lab; Regression Analysis}
\end{center}
\vspace{0.5cm}
\begin{flushright}
\fontsize{14pt}{12pt} \selectfont
Hong Gildong \\
Department of Statistics \\
The Graduate School \\
Seoul National University
\end{flushright}
\vspace{0.5cm}

\noindent
This is for English abstract. This is for English abstract.This is for English abstract.This is for English abstract.This is for English abstract.This is for English abstract.This is for English abstract.This is for English abstract.This is for English abstract.This is for English abstract.This is for English abstract.This is for English abstract.This is for English abstract.This is for English abstract.This is for English abstract.This is for English abstract.This is for English abstract.This is for English abstract.This is for English abstract.This is for English abstract.This is for English abstract.This is for English abstract.This is for English abstract.This is for English abstract.This is for English abstract.This is for English abstract.This is for English abstract.This is for English abstract.This is for English abstract.This is for English abstract.This is for English abstract.This is for English abstract.This is for English abstract.This is for English abstract.This is for English abstract.This is for English abstract.

\end{abstract}

\tableofcontents
\listoffigures
\listoftables

\cleardoublepage
\pagenumbering{arabic}

% \chapter{Introduction}
% Introduction.
%
% \chapter{...}
% ...
%
% \chapter{Conculsion}
% Conculsion.

\chapter{Introduction}\label{introduction}

It would be much clean to white thesis using markdown. e.g., there are
some \href{https://github.com}{github} repositories dedicated for this.
\url{https://github.com/tompollard/phd_thesis_markdown} is an example.

\begin{figure}
\begin{center}
\includegraphics[width=\textwidth]{figs/Octocat}
\end{center}
\caption{A cute Octocat. Yes, we are on GitHub!!!}
\end{figure}

\begin{table}
\caption{A simple table.}
\begin{center}
\begin{tabular}{ l | c | r }
  1 & 2 & 3 \\
  4 & 5 & 6 \\
  7 & 8 & 9 \\
\end{tabular}
\end{center}
\end{table}

\chapter{A Great Model}\label{a-great-model}

\section{Something even better}\label{something-even-better}

\subsection{Even a better one?}\label{even-a-better-one}

\begin{itemize}
\tightlist
\item
  It is easy to make a list in markdown.
\item
  seriously!
\end{itemize}

\begin{enumerate}
\def\labelenumi{\arabic{enumi}.}
\tightlist
\item
  Even a numbered list.
\item
  don't ya think?
\end{enumerate}

\begin{table}
\caption{Just another table.}
\begin{center}
\begin{tabular}{ l | c | r }
  1 & 2 & 3 \\
  4 & 5 & 6 \\
  7 & 8 & 9 \\
\end{tabular}
\end{center}
\end{table}

\chapter{Conclusion}\label{conclusion}

Argh. Why am I doing this? \citet{Nesterov:DoklAkadNaukSssr:1983}
\citep{Boyd:ConvexOptimization:2004}
\citet{Bertsekas:ConvexOptimizationTheory:2009}

\addcontentsline{toc}{chapter}{\bibname}

\bibliography{references}

\appendix

\chapter{How to use Appendix}\label{how-to-use-appendix}

Just add appendix here! After peeking through many of theses, it was
clear that most of the students are likely to have appendices in their
thesis. This causes some messy block like

\begin{Shaded}
\begin{Highlighting}[]
\NormalTok{<!--}\CommentTok{ DO NOT REMOVE THE FOLLOWING!!!!!!!! -->}
\NormalTok{<!--}\CommentTok{ references. -->} 
\NormalTok{<!--}\CommentTok{ note: this is how to use comments }
\CommentTok{in a markdown file.-->}
\NormalTok{<!--}\CommentTok{ I know it's not clean...  -->}
\NormalTok{\textbackslash{}addcontentsline\{toc\}\{chapter\}\{\textbackslash{}bibname\}}
\NormalTok{\textbackslash{}bibliography\{references\} <!--}\CommentTok{set to correct bibliography file name -->}

\NormalTok{\textbackslash{}appendix}
\NormalTok{<!--}\CommentTok{ DO NOT REMOVE THE ABOVE!!!!!!!! -->}
\end{Highlighting}
\end{Shaded}

in the middle of the markdown, but I could not find a better way to use
appendices than this.

\chapter{Dang it}\label{dang-it}

dszf







\keywordalt{갑, 을, 병, 정}
\begin{abstractalt}
국문초록국문초록국문초록국문초록국문초록국문초록국문초록국문초록국문초록국문초록국문초록국문초록국문초록국문초록국문초록국문초록국문초록국문초록국문초록국문초록국문초록국문초록국문초록국문초록국문초록국문초록국문초록국문초록국문초록국문초록국문초록국문초록국문초록국문초록국문초록국문초록국문초록국문초록국문초록국문초록국문초록국문초록국문초록국문초록국문초록국문초록국문초록국문초록묵문초록국문초록국문초록국문초록국문초록국문초록국문초록국눈초록국문초록국문초록국문초록국문초록국문초록국문초록국문초록국문초록국문초록국문초록국문초록국문초록국문초록국문초록국문초록국문초록국문초록국문초록국문초록국문초록국문초록국문초록국문초록국문초록국문초록국문초록국문초록국문초록국문초록국문초록국문초록국문초록국문초록국문초록국문초록국문초록국문초록국문초록국문초록국문초록국문초록국문초록궁문초록국문초록국문초록국문초록국문초록국문초록국문초록국문초록국문초록국문초록국문초록국문초록국문초록국문초록국문초록국문초록국문초록국문초록국문초록국문초록국물초록국문초록

\end{abstractalt}

\acknowledgement
\input{acknowledgement.tex}


\end{document}

