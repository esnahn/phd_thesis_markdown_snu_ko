\renewcommand{\abstractname}{국문초록} %
%\renewcommand{\alsoname}{see also} % (makeidx package)
\renewcommand{\appendixname}{부록} %
\renewcommand{\bibname}{참고문헌} % (report,book)
%\renewcommand{\ccname}{cc} % (letter)
%\renewcommand{\chaptername}{Chapter} % (report,book)
\renewcommand{\contentsname}{목차} %
%\renewcommand{\enclname}{encl} % (letter)
\renewcommand{\figurename}{그림} % (for captions)
%\renewcommand{\headtoname}{To} % (letter)
\renewcommand{\indexname}{색인} %
\renewcommand{\listfigurename}{그림 목차} %
\renewcommand{\listtablename}{표 목차} %
%\renewcommand{\pagename}{Page} % (letter)
%\renewcommand{\partname}{Part} %
%\renewcommand{\refname}{References} % (article)
%\renewcommand{\seename}{see} % (makeidx package)
\renewcommand{\tablename}{표} % (for caption)
\newcommand{\keywordname}{주요어}
\newcommand{\keywordnamealt}{Keywords}
\newcommand{\studentnumbername}{학번}
\newcommand{\studentnumbernamealt}{Student Number}
\newcommand{\abstractnamealt}{Abstract}
\newcommand{\acknowledgementname}{감사의 글}

% %% maketitle 명령 제거.
% \let\oldmaketitle=\maketitle
% \renewcommand\maketitle{}

%%% 줄 간격: header.tex의 setspace 활성화할 것
%%% 기본값이 가장 낫다...
% \setstretch{1.7}
%%% 1줄 좀 넘게 띄기 (소스 참조)
% \onehalfspacing

%%% 서울대 논문 글씨 크기
\newcommand{\snularge}{\fontsize{14}{18}\selectfont}
\newcommand{\snuhuge}{\fontsize{16}{18}\selectfont}
\newcommand{\snutitleHuge}{\fontsize{22}{22}\selectfont}

%%% pandoc 번호 있는 장은 \chapter 번호 없는 장은 \chapter* \addcontentsline
%%% 추가로 모양 없는 장 만들기

\newcommand{\snunoheaderchapter}[1]{%
	\newpage
	\begin{center}%
		\snuhuge\textbf{#1}
	\end{center}
	\addcontentsline{toc}{chapter}{#1}
}





% \addcontentsline{toc}{chapter}{\abstractname}
% {%
%     \centering
%     \normalfont
%     \linespread{1.0}
%     \fontsize{16pt}{16pt}\selectfont
%     \textbf{\abstractname}\par%\nobreak
% }

% \def\@makechapterhead#1{%
%   \vspace*{50\p@}%
%   {\parindent \z@ \raggedright \normalfont
%     \ifnum \c@secnumdepth >\m@ne
%         \huge\bfseries \@chapapp\space \thechapter
%         \par\nobreak
%         \vskip 20\p@
%     \fi
%     \interlinepenalty\@M
%     \Huge \bfseries #1\par\nobreak
%     \vskip 40\p@
%   }}
% \def\@schapter#1{\if@twocolumn
%                    \@topnewpage[\@makeschapterhead{#1}]%
%                  \else
%                    \@makeschapterhead{#1}%
%                    \@afterheading
%                  \fi}
% \def\@makeschapterhead#1{%
%   \vspace*{50\p@}%
%   {\parindent \z@ \raggedright
%     \normalfont
%     \interlinepenalty\@M
%     \Huge \bfseries  #1\par\nobreak
%     \vskip 40\p@
%   }}

% %% 감사의 글
% \newcommand{\acknowledgement}{%
% 	\if@snu@ko
% 		\if@openright \cleardoublepage
% 		\else \clearpage \fi
% 		\@afterindentfalse
% 		\phantomsection
% 		\addcontentsline{toc}{chapter}{\acknowledgementname}
% 		{%
% 			\centering
% 			\normalfont
% 			\linespread{1.0}
% 			\fontsize{16pt}{16pt}\selectfont
% 			\textbf{\acknowledgementname}\par\nobreak
% 		}
% 		\vspace{2cm}
% 		\@afterheading
% 	\else
% 		\chapter*{\acknowledgementname}
% 		\@mkboth{\MakeUppercase\acknowledgementname}%
% 			{\MakeUppercase\acknowledgementname}%
% 		\addcontentsline{toc}{chapter}{\acknowledgementname}%
% 	\fi
% }
























