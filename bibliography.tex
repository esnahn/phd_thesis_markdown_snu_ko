\begin{thebibliography}{00}
\addcontentsline{toc}{chapter}{\bibname}

% 영문저널의 경우
    \bibitem{ref1} B. Jeon and J. Jeong, ``Blocking artifacts
    reduction in image compression with block boundary discontiunity
    criterion,'' {\em IEEE Transactions on Circuits and Systems for
    Video Tech.}, vol. 8, no.3, pp. 345-357, June 1998.

% 영문학술대회의 경우
    \bibitem{ref2} W. G. Jeon and Y. S. Cho, ``An equalization
    technique for OFDM and MC-CDMA in a multipath fading channels,''
    in {\em Proceedings of IEEE Conference on Acoustics, Speech and
    Signal Processing}, Munich, Germany, May 1997. pp. 2529-2532.

% 국내저널의 경우
    \bibitem{ref3} 김남훈, 정영철, ``평탄한 통과대역 특성을 갖는
    새로운 구조의 광도 파로열 격자 라우터,'' {\em 전자공학회논문지},
    제35권 D편, 제3호, 56-62쪽, 1998년 3월.

% 국내학술대회의 경우
    \bibitem{ref4} 윤남국, 김수종, ``무선 센서 네트워크에서의 에너지
    효율적인 그라디언트 기반 라우팅 기법,'' {\em 한국정보과학회
    2006년 추계학술대회}, 제12권, 제2호, 2006년 10월. pp.
    1372-1374.

% 단행본의 경우
    \bibitem{ref5} C. Mead and L. Conway, {\em Introduction to VLSI
    Systems}, Addison-Wesley, Boston, 1994.

% URL
    \bibitem{ref6} The SolarMESH Network,
    http://owl.mcmater.ca/solarmesh

% Technical Report의 경우
    \bibitem{ref7} K. E. Elliott and C. M. Greene, ``A local adaptive
    protocol,'' Argonne National Laboratory, Argonne, France,
    Technical Report 916-1010-BB, 1997.

% 학위논문의 경우
    \bibitem{ref8} T. Kim, ``Scheduling and Allocation Problems in
    High-level Synthesis,'' Ph. D. Dissertation, ECE Department,
    Univ. of Illinois at U-C, 1993.

% 특허의 경우
    \bibitem{ref9} Sunghyun Choi, ``Wireless MAC protocol based on a
    hybrid combination of slot allocation, token passing, and
    polling for isochronous traffic,'' U.S. Patent No. 6,795,418,
    September 21, 2004.

% 표준
    \bibitem{ref10} IEEE Std. 802.11-1999, Part 11: Wireless LAN
    Medium Access Control (MAC) and Physical Layer (PHY)
    specifications, Reference number ISO/IEC 8802-11:1999(E), IEEE
    Std. 802.11, 1999 edition, 1999.

\end{thebibliography}
